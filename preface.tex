\chapter{Preface}
\begin{fullwidth}
	"I want to know something about everything after I start my career as a system engineer" said Caitlyn when we were chatting while doing a convection experiment in ME320 heat transfer lab. Her words somehow tug at my heartstrings as that was one of my goals when I started my journey as a naive Ph.D. student. After 5 years into my Ph.D. program, I am still not confident that I know enough to know everything. My failure of growing into a knowledgeable researcher might be attributed to my consistent tendency of being attracted by random things. I still waste half of my day sometimes skimming through previews of science/math books on amazon just because I found the titles suit my mood. 
	
	It was my job as a lab TA that forced me to unite all the random things in my mind, and recast them into complete stories. My original plan was just to make one or two notes for students who are not able to attend my sessions due to their exposure to COVID. In order to keep myself focused during the writing process, some impractical thought experiments are inevitably added in the first two notes. Unexpectedly, I found myself learning new things organically by developing "stories" in the notes. Things that were never mentioned in my undergrad/Ph.D. classes suddenly become indispensable to deliver vivid explanations of physical phenomena. That was the moment when I decided to finish the notes for all the labs, and see what else I can bring to the table where people are bored of learning things off textbooks. 
	
	The notes shown here defines the best part of me both as a learner and as a researcher. Writing them is a journey of self-redemption during which I proved, at least to myself, that I might be able to derive and create cool things by using simple mathematical language and intuitions of my own. I want to give my special thinks to some of my committed students: Rena Kanegae, Caitlyn Peters, Stefan Kamzol, Ashish Mittal, Haley Middendorf, and Nathan Asad for their thought-provoking questions and suggestions to the labs. And I will never forget the kind comments from Dr. Hongliang Qian and my buddy Arjun Sanjay Goswami.
	\begin{flushright}
		--Sizhe Liu, \\UIUC
	\end{flushright}
\end{fullwidth}
